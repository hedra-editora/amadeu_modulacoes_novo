\textbf{A sociedade de controle} parte de alguns conceitos de Gilles Deleuze, em continuação com a ideia de \textit{sociedade disciplinar} de Michel Foucault, para pensar uma interface inédita das sociedades contemporâneas: a modulação de comportamentos, opiniões, gostos e tendências no ambiente digital. Diferente do controle ou da disciplina, a modulação conduz seus usuários por caminhos oferecidos pelos dispositivos algorítmicos que gerenciam os
interesses de influenciadores e influenciados. Os seis artigos aqui reunidos dão uma contribuição original e crítica para pensar a emergente esfera da modulação nas redes digitais.

\textbf{Joyce Souza} é jornalista e cientista social. Doutoranda em Ciências Humanas e Sociais na Universidade Federal do \textsc{abc} (\textsc{ufabc}), pesquisadora do Laboratório de Tecnologias Livres (LabLivre/\,\textsc{ufabc}) e coprodutora do podcast \textit{Tecnopolítica}. É também membra do Movimento das Tecnologias Não Alinhadas (\textsc{natm}). 

\textbf{Rodolfo Avelino} é doutor pelo Programa de Ciências Humanas e Sociais (\textsc{pchs}) da Universidade Federal do \textsc{abc} (\textsc{ufabc}), mestre em \textsc{tv} Digital pela Unesp Bauru e bacharel em Sistemas de Informação pelo Centro Universitário Fundação Santo André. É também professor do Insper e pesquisador no Laboratório de Tecnologias Livres (LabLivre/\,\textsc{ufabc}), além de componente da diretoria da \textsc{ong} Coletivo Digital, onde atualmente realiza cursos e eventos relacionados à privacidade e à proteção de dados em ambientes computacionais.

\textbf{Sérgio Amadeu da Silveira} é sociólogo e doutor em Ciência Política pela Universidade de São Paulo (2005). É professor associado da Universidade Federal do \textsc{abc} (\textsc{ufabc}). Publicou \textit{Comunicação digital e a construção dos commons} (Perseu Abramo, 2007), \textit{Tudo sobre tod@s: Redes digitais, privacidade e venda de dados pessoais} (Edições Sesc, 2017), \textit{Democracia e os códigos invisíveis: como os algoritmos estão modulando comportamentos e escolhas políticas} (Edições Sesc, 2019), entre outros.

